\chapter{Introducción}

El propósito de esta investigación consiste en desarrollar un brazo robótico colaborativo de coste accesible para que las pequeñas y medianas empresas puedan aumentar la eficiencia y calidad en sus procesos así como librar a sus trabajadores de tareas repetitivas o potencialmente peligrosas. %This was five.

Aún con el crecimiento en la demanda de los robots colaborativos, el precio de éstos no ha sufrido grandes cambios desde su lanzamiento, pues su precio promedio ronda en los \textbf{\$50,000 dólares}, lo cual es prohibitivo para las pequeñas y medianas empresas. % This was four.

Los robots colaborativos, también denominados \textit{cobots}, son aquellos que permiten a los humanos ocupar la misma área de trabajo que éstos y ofrecer la interacción segura entre robot y humano con el fin de realizar una tarea común.  % This was the first.

Los \textit{cobots} ofrecen mucho más flexibilidad en su operación con respecto a los robots industriales tradicionales. Las desventajas consisten en sacrificar la carga útil máxima así como la  eficiencia y el alcance de los robots industriales tradicionales. En resumen, los robots colaborativos son un excelente compromiso entre el trabajo manual y la automatización industrial. \cite{Zaatari2019} % This was second.

El interés en los robots colaborativos ha ido en aumento en la última década, es por esto que los grandes fabricantes de robots como ABB, KUKA o Universal Robots han desarrollado productos para este nicho de mercado. Por su parte, las grandes emprezas de manufactura como Audi, Volkswagen y Nissan han incorporado robots colaborativos en sus lineas de ensamblaje. \cite{Zaatari2019} % And third.

\begin{comment}
Párrafo uno: Introducción sobre los robots colaborativos
Párrafo dos: Ventajas con respecto a robots tradicionales
Párrafo tres: Pertinencia de la investigación en la actualidad
Párrafo cuatro: Propósito de la investigación.
Párrafo cinco: Propósito de la investigación e interés social.
\end{comment}


\section{Objetivos}

El objetivo general de esta investigación consiste en diseñar y construir un brazo robótico colaborativo de seis grados de libertad para sistemas de manufactura flexible.

Los objetivos específicos se mencionan a continuación:

\begin{itemize}
\item Desarrollar el modelo matemático cinemático y dinámico de un brazo robótico de seis grados de libertad.
\item Construir un brazo robótico de coste accesible y fácil fabricación, así como compartir su diseño y componentes con una licencia de código abierto. 
\item Optimizar el modelado de piezas para su correcta manufactura con máquinas de manufactura aditiva.
\item Desarrollar ¿utilizar? una metodología apegada a la ingeniería de sistemas basadas en modelos.
\end{itemize}

\section{Justificación}

El desarrollo de un brazo robótico de seis grados de libertad para sistemas de manufactura flexible que se pretende realizar es relevante en diversos ámbitos del conocimiento. 

Por último, al estar pensado como un desarrollo de código abierto, contribuirá al acervo tecnológico de la humanidad y podrá ser copiado, modificado y mejorado alrededor del mundo. 


\section{Hipótesis}

