\chapter{Introducción}

Los robots colaborativos, también denominados \textit{cobots}, son aquellos que permiten a los humanos ocupar la misma área de trabajo que éstos y ofrecer la interacción segura entre robot y humano con el fin de realizar una tarea común.  % This was the first.

Los \textit{cobots} ofrecen mucho más flexibilidad en su operación con respecto a los robots industriales tradicionales. Las desventajas consisten en sacrificar la carga útil máxima así como la  eficiencia y el alcance de los robots industriales tradicionales. En resumen, los robots colaborativos son un excelente compromiso entre el trabajo manual y la automatización industrial. \cite{Zaatari2019} % This was second.

El interés en los robots colaborativos ha ido en aumento en la última década, es por esto que los grandes fabricantes de robots como ABB, KUKA o Universal Robots han desarrollado productos para este nicho de mercado. Por su parte, las grandes emprezas de manufactura como Audi, Volkswagen y Nissan han incorporado robots colaborativos en sus lineas de ensamblaje. \cite{Zaatari2019} % And third.

Aún con el crecimiento en la demanda de los robots colaborativos, el precio de éstos no ha sufrido grandes cambios desde su lanzamiento, pues su precio promedio ronda en los \textbf{\$50,000 dólares}, lo cual es prohibitivo para compañías en países emergentes. % This was four.

El propósito de esta investigación consiste en desarrollar un brazo robótico colaborativo de coste accesible para que las pequeñas y medianas empresas puedan aumentar la eficiencia y calidad en sus procesos así como librar a sus trabajadores de tareas repetitivas o potencialmente peligrosas. %This was five.



\begin{comment}
Párrafo uno: Introducción sobre los robots colaborativos
Párrafo dos: Ventajas con respecto a robots tradicionales
Párrafo tres: Pertinencia de la investigación en la actualidad
Párrafo cuatro: Propósito de la investigación.
Párrafo cinco: Propósito de la investigación e interés social.
\end{comment}

\section{Planteamiento del problema}

En el campo de la robótica y la automatización existe una brecha pronunciada entre la automatización industrial disponible para las grandes empresas de manufactura y lo que las pequeñas y medianas empresas se pueden costear.  

Si bien es cierto que la demanda de brazos robóticos en la industriaen todo el mundo crece año con año  \cite{summary2019}, las innovaciones en este campo particular han sido pocas, y al no existir innovación la competencia se centra en unos cuantas empresas.

Recientemente, un nuevo tipo de brazos robóticos han llegado al mercado, denóminados Cobots o robots colaborativos, los cuales, su principal función es el poder trabajar lado a lado con humanos sin poner en riesgo la seguridad de ambos.

Aún con estos nuevos desarrollos, el precio de los brazos robóticos colaborativos es prohibitivo para empresas pequeñas, por lo que, en este documento se trata de desarrollar un brazo robótico colaborativo de código abierto y de coste accesible.



\section{Objetivos}

El objetivo general de esta investigación consiste en diseñar y construir un brazo robótico colaborativo de seis grados de libertad para sistemas de manufactura flexible. Los objetivos específicos se mencionan a continuación:

\begin{itemize}
\item Desarrollar el modelo matemático cinemático y dinámico de un brazo robótico de seis grados de libertad.
\item Construir un brazo robótico de coste accesible y fácil reproducción.
\item Optimizar el modelado de piezas para su correcta manufactura con máquinas de manufactura aditiva.
\item Utilizar una metodología apegada a la ingeniería de sistemas basadas en modelos.
\item Compartir su diseño, componentes y software con una licencia de código abierto. 
\end{itemize}

\section{Justificación}

El desarrollo de un brazo robótico colaborativo de seis grados de libertad para sistemas de manufactura flexible que se pretende realizar es relevante en diversos ámbitos del conocimiento, tales como la robótica, la inteligencia artifical y la metodología de ingeniería de sistemas basado en modelos.

Para lograr la descentralización deseada en la industria 4.0 es necesario que los medios de producción sean democráticos y estén al alcance de la mayor cantidad de personas posibles, es por esto que es necesario desarrollar tecnología de coste accesible y de código abierto.

El elevado precio de los robots colaborativos actuales se basa en diferentes factores, entre los que se encuentran:

\begin{itemize}
\item Certificaciones de seguridad, resistencia contra los elementos o seguridad ambiental.
\item Accesorios incluidos tales como la consola de control y aprendizaje o elementos de anclaje.
\item Elementos protegidos por patentes o excesivamente costosos, como los \textit{harmonic drive}.
\item Investigación y desarrollo.
\end{itemize}

Es posible eliminar la mayoría de estos costos al desarrollar únicamente una plataforma mecánica y de software robusta sobre la cual la comunidad pueda desarrollar, remplazando los elementos costosos por alternativas más baratas y eliminando los accesorios y las certificaciones.

Otro punto importante a tener en cuenta, es que la naturaleza de este proyecto es servir como plataforma para el desarrollo de la investigación, pues con ayuda del brazo robótico se podrá desarrollar invesitigación en el área de inteligencia artifical, tales como aprendizaje reforzado o visión computacional.

Por último, al estar pensado como un desarrollo de código abierto, contribuirá al acervo tecnológico de la humanidad y podrá ser replicado, modificado y mejorado alrededor del mundo. 


\section{Hipótesis}

Es posible realizar un brazo robótico colaborativo de coste accesible para implementarlo en sistemas de manufactura flexible. Así mismo, es posible realizar la mayoría de los componentes estructurales por medio de manufactura aditiva, en especial los reductores de velocidad necesarios para alcanzar los requisitos de carga útil máxima y velocidad de operación.

