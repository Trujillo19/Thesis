\chapter{Introducción}

Aquí empieza el viaje	

\section{Objetivos}

El objetivo general de esta investigación consiste en diseñar y construir un brazo robótico de seis grados de libertad para sistemas de manufactura flexible.

Los objetivos específicos se mencionan a continuación:

\begin{itemize}
\item Desarrollar el modelo matemático cinemático y dinámico de un brazo robótico de seis grados de libertad.
\item Construir un brazo robótico de coste accesible y fácil fabricación, así como compartir su diseño y componentes con una licencia de código abierto. 
\item Optimizar el modelado de piezas para su correcta manufactura con máquinas de manufactura aditiva.
\end{itemize}

\section{Justificación}

El desarrollo de un brazo robótico de seis grados de libertad para sistemas de manufactura flexible que se pretende realizar es relevante en diversos ámbitos del conocimiento. 

\begin{comment}
De acuerdo con Spong et al. \cite[p.~1]{Spong2005}, la mayoría de las aplicaciones de la robótica se centran en brazos robóticos industriales que operan en fabricas con entornos estructurados, sin embargo, a medida que la competencia en la industria se vuelve más intensa, se vuelve necesaria una nueva estrategia de manufactura, llamada sistema de manufactura flexible, en
\end{comment}

En el ámbito educativo, un brazo robótico de código abierto puede ser utilizado como elemento de estudio o réplica, así como para la realización de prácticas o trabajos posteriores sobre el mismo.



\section{Hipótesis}

