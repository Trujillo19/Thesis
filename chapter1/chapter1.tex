\chapter{Introducción}

Los robots colaborativos, también denominados \textit{cobots}, son aquellos que permiten a los humanos ocupar la misma área de trabajo que éstos y ofrecer la interacción segura entre robot y humano con el fin de realizar una tarea común. 

Los \textit{cobots} ofrecen mucho más flexibilidad en su operación con respecto a los robots industriales tradicionales, por lo que es posible su utilización en la manufactura de productos personalizados en masa. Las desventajas consisten en sacrificar la carga útil máxima así como la  eficiencia y exactitud de los robots industriales tradicionales. En resumen, los robots colaborativos se encuentran en el punto medio entre el trabajo manual y la automatización industrial. \cite{Zaatari2019}

El interés en los robots colaborativos ha ido en aumento en la última década, es por esto que los grandes fabricantes de robots como ABB, KUKA o Universal Robots han lanzado productos que satisfacen esta necesidad. Por su parte, las grandes emprezas de manufactura como Audi, Volkswagen y Nissan han incorporado robots colaborativos en sus lineas de ensamblaje \cite{Zaatari2019}



\begin{comment}
La finalidad de esta debe ser suministrar suficientes ante-
cedentes para que el lector pueda comprender y evaluar los resultados del

estudio sin necesidad de consultar publicaciones anteriores sobre el tema.
Debe presentar también el fundamento racional del estudio. Por encima de
todo, hay que manifestar breve y claramente cuál es el propósito al escribir
el artículo. Hay que elegir las referencias cuidadosamente para suministrar

\end{comment}


\section{Objetivos}

El objetivo general de esta investigación consiste en diseñar y construir un brazo robótico colaborativo de seis grados de libertad para sistemas de manufactura flexible.

Los objetivos específicos se mencionan a continuación:

\begin{itemize}
\item Desarrollar el modelo matemático cinemático y dinámico de un brazo robótico de seis grados de libertad.
\item Construir un brazo robótico de coste accesible y fácil fabricación, así como compartir su diseño y componentes con una licencia de código abierto. 
\item Optimizar el modelado de piezas para su correcta manufactura con máquinas de manufactura aditiva.
\item Desarrollar ¿utilizar? una metodología apegada a la ingeniería de sistemas basadas en modelos.
\end{itemize}

\section{Justificación}

El desarrollo de un brazo robótico de seis grados de libertad para sistemas de manufactura flexible que se pretende realizar es relevante en diversos ámbitos del conocimiento. 

Por último, al estar pensado como un desarrollo de código abierto, contribuirá al acervo tecnológico de la humanidad y podrá ser copiado, modificado y mejorado alrededor del mundo. 


\section{Hipótesis}

