\chapter{Marco teórico}

\section{Brazos robóticos industriales}

De acuerdo con Spong et al. \cite{Spong2005}, la mayoría de las aplicaciones en el campo de la robótica se centran en brazos robóticos industriales que operan en fabricas con entornos estructurados, es por esto su gran importancia.

La Federación Internacional de Robótica ha reportado un crecimiento promedio anual a partir del 2013 del 19\% en la demanda de brazos robóticos, siendo la industria automotríz la dominante, seguida por la industria eléctrica y electrónica. \cite{summary2019}

Un brazo robótico puede ser modelado como una cadena de eslabones rígidios conectados entre si por articulaciones flexibles \cite{Schilling2001}. Como su nombre lo indica, el diseño es similar a un brazo humano, con los eslabones representando el brazo y el antebrazo, mientras que las articulaciones representan el hombro, codo y muñeca.

Los primeros 

\subsection{Clasificación de los robots}

La clasificación de los brazos robóticos tiene dos grandes ramas, una de estas es por la fuente de su energía y la otra por la geometría de trabajo. 

La fuente de energía puede ser eléctrica o hidráulaca, la mayoría de los manipuladores hoy en día usan servomotores o motores a pasos de corriente continua \cite{Schilling2001}.

Dependiendo de la geometría de trabajo, los brazos robóticos  se dividen en: cartesianos, cilíndricos, esféricos, SCARA o articulados.

\section{Robótica colaborativa}

La robótica colaborativa surge como una rama de la robótica dispuesta a hacer más fácil el trabajo en lineas de producción, así como aliviar problemas en la espalda relacionados con tareas de ensamblado final en posiciones no ergonómicas \cite{cobot2018}\cite{cobotreview}.



\section{Sistemas de manufactura flexible}
\section{Ingeniería de sistemas basado en modelos}
