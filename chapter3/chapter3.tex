\chapter{Metodología}

\section{Modelos matemáticos}

De acuerdo con \cite{Dombre2007}, el diseño y control de robots requiere diversos modelos matemáticos, tales como:

\begin{itemize}
\itemsep0em 

\item Cinemática directa e inversa, es decir, encontrar la posición del efector final en términos de las coordenadas de las articulaciones y viceversa.
\item Cinemática de la velocidad, encontrar la velocidad del efector final en términos de la velocidad de las articulaciones y viceversa.
\item Modelo dinámico, el cual establece la relación entre los torques o fuerzas que ejercen los actuadores y las posiciones, velocidades y aceleraciones de las articulaciones.
\end{itemize}

\begin{figure}
    \centering
    \includegraphics[scale=0.7]{./img/chapter3/robotarmprototype.png}
    \caption{Boceto del brazo robótico propuesto}
    \label{fig:roboticarmprototype}
\end{figure}

En este capítulo se desarrollarán los modelos matemáticos necesarios para simular y diseñar el brazo robótico, así como predecir el comportamiento del mismo.

Para realizar estos modelos, es importante contar con los parámetros físicos y geométricos del robot, los cuales aún no están completamente definidos, por lo que, para una primera aproximación, se utilizarán valores experimentales. Éstos se mencionan en la tabla \ref{table:parametrosbrazorobotico}.

Otros parámetros necesarios para el desarrollo del modelo matemático del brazo robótico son el alcance total del brazo, el cual deberá ser de mínimo 500 mm, la velocidad, la cuál deberá estar en un rango entre 5 RPM y 30 RPM, y por último, la carga útil, la cuál deberá ser de 2 kg.

\begin{table}
\centering
\caption{Parámetros del brazo robótico}
 \label{table:parametrosbrazorobotico}
\begin{tabular}{l|l|l|l|l|l|l|}
\textbf{Eslabones}                       &  1   &  2   &  3   &  4   &  5   &  6    \\ 
\hline
Longitud           & 0.152       & 0.104       & 0.244       & 0.104       & 0.213       & 0.104        \\
Masa               & 0           & 2           & 1           & 0.8         & 0.8         & 0.2          \\
Centro de gravedad & {[}0, 0, 0] & {[}0, 0, 0] & {[}0, 0, 0] & {[}0, 0, 0] & {[}0, 0, 0] & {[}0, 0, 0]  \\
Matriz de inercia      & {[}0, 0, 0] & {[}0, 0, 0] & {[}0, 0, 0] & {[}0, 0, 0] & {[}0, 0, 0] & {[}0, 0, 0]  \\
Fricción en eslabón    & 0.00148     & 0.00817     & 0.00138     & 7.12e-05    & 8.26e-05    & 3.67e-05     \\
Fricción de Coulomb    & 0.395       & 0.126       & 0.132       & 0.0113      & 0.00926     & 0.00396      \\
Inercia del motor      & 0.002       & 0.002       & 0.002       & 3.3e-05     & 3.3e-05     & 3.3e-05     
\end{tabular}
\end{table}

En la figura \ref{fig:roboticarmprototype} podemos ver un boceto del brazo robótico que se planea implementar.

Con estos datos definidos, es posible empezar la realización de los modelos matemáticos.

\subsection{Cinemática directa e inversa.}

En esta sección se desarrollará el primer modelo matemático del brazo robótico, la cinemática directa e inversa, la cual se trata de encontrar la posición final del brazo dependiendo de las coordenadas $\theta$ de sus articulaciones o, para el caso de la cinemática inversa, calcular las coordenadas $\theta$ necesarias para llegar a una posición $(x,y,z)$ dada.

\subsubsection{Cinemática directa} 
Como ya se mencionó anteriormente, la cinemática directa de un brazo robótico se refiere al cálculo de la posición y orientación del marco de referencia del efector final a partir de las coordenadas $\theta$ de sus articulaciones. \cite{University2017}

Un ejemplo de esto lo podemos encontrar en la figura \ref{fig:forwardkinematic2}, donde se analiza la cinemática directa de un brazo robótico de tres grados de libertad. En este ejemplo podemos apreciar que la longitud de los eslabones se representan como $L_1, L_2, L_3$ respectivamente y se ha escogido un marco de referencia fijo denominado \{0\}, mientras que el marco de referencia del efector final es \{4\}. \cite{University2017}

Entonces, para conocer la posición final es necesaria la siguente ecuación:
\begin{equation}
\label{eq:cinematicadirecta1}
x = L_1 \cos(\theta_1)+L_2 \cos(\theta_1 + \theta_2)+ L_3 \cos(\theta_1 + \theta_2 + \theta_3)
\end{equation}
\begin{equation}
\label{eq:cinematicadirecta2}
y = L_1 \sin(\theta_1)+L_2 \sin(\theta_1 + \theta_2)+ L_3 \sin(\theta_1 + \theta_2 + \theta_3)
\end{equation}
\begin{equation}
\label{eq:cinematicadirecta3}
\phi = \theta_1 + \theta_2 + \theta_3
\end{equation}

Las ecuaciones \ref{eq:cinematicadirecta1}, \ref{eq:cinematicadirecta2} y \ref{eq:cinematicadirecta3} se les conoce como \textbf{ecuación de cinemática directa} y aunque es válida, se vuelve rápidamente compleja para un brazo robótico de seis grados de libertad, por esta razón, el enfoque sistemático de resolución es utilizar matrices de transformación homogenea y la convención de Denavit-Hartenberg.

\begin{figure}
    \centering
    \includegraphics[scale=0.6]{./img/chapter3/forwardkinematic.png}
    \caption{Cinemática directa de un brazo robótico de 3 grados de libertad \cite{University2017}}
    \label{fig:forwardkinematic2}
\end{figure}

\paragraph{Matriz de transformación homogénea.}

Una matriz de transformación homogenea no es nada más que la representación matricial de un movimiento rígido \cite{Spong2005}, así, la posición y orientación del efector final puede ser fácilmente calculada utilizando multiplicación de matrices.

La forma más general de una matriz de transformación homogenea puede ser escrita de la siguiente manera:

\begin{equation}
T^0_1 =
\begin{bmatrix}
n_x & s_x & a_x & d_x\\
n_y & s_y & a_y & d_y\\
n_z & s_z & a_z & d_z\\
0 & 0 & 0 & 1
\end{bmatrix} = 
\begin{bmatrix}
n & s & a & d\\
0 & 0 & 0 & 1
\end{bmatrix}
\end{equation}

En dónde $n$ es un vector representando la dirección $x_1$ en el sistema $o_0 x_0 y_0 z_0$, de manera similar, $s$ representa la dirección de $y_1$ y $a$ representa la direción de $z_1$. Por su parte, $d$ es un vector representando la posición desde el origen $o_0$ hasta el origen $o_1$ en el marco de referencia $o_0 x_0 y_0 z_0$.


Según \cite{University2017}, existen tres usos principales para una matriz de transformación homogénea:

\begin{enumerate}
\itemsep0em
  \item Para representar la configuración (posición y orientación) de un cuerpo rígido.
  \item Para cambiar el marco de referencia en el cuál está representado un vector u otro marco de referencia.
  \item Para desplazar un vector o un marco de referencia.
\end{enumerate}

Para el caso que nos ocupa, necesitamos la matriz de transformación homogénea desde la base fija del robot hasta su efector final, descrita con la ecuación siguiente:
\begin{equation}
\label{eq:homogeneustransformationmatrix}
\begin{split}
{}_{0}^{7}T = \mathscr{T}_z(a_1)\oplus\mathscr{R}_z(\theta_1)\oplus\mathscr{T}_x(b_1)\oplus\mathscr{R}_x(\theta_2)\oplus\mathscr{T}_z(c_1)\oplus\mathscr{R}_x(\theta_3) \\ \oplus\mathscr{T}_z(d_1)\oplus\mathscr{T}_x(d_2)\oplus\mathscr{R}_x(\theta_4)\oplus\mathscr{T}_x(e_1)\oplus\mathscr{R}_z(\theta_5)\oplus\mathscr{T}_z(f_1)\oplus\mathscr{R}_x(\theta_6)
\end{split}
\end{equation}

Dónde $a_1, b_1, c_1, d_1, d_2, e_1$ y $f_1$ son las longitudes de los eslabones del brazo robótico. 

\begin{figure}
    \centering
    \includegraphics[scale=0.6]{./img/chapter3/KinematicDiagramML.png}
    \caption{Cadena cinemática}
    \label{fig:kinematicchain}
\end{figure}


En la imagen \ref{fig:kinematicchain} podemos observar la cadena cinemática de nuestro brazo robótico, fue creada con un algoritmo en MATLAB con ayuda de la herramienta Robotic Toolbox \cite{Corke2017}, dicho código puede consultarse en el Anexo 1.


\paragraph{Convención de Denavit-Hartenberg.} En muchas aplicaciones, es necesario representar los parámetros cinemáticos de forma simplificada con ayuda de la notación Denavit-Hartenberg, con esta convención se simplifa considerablemente el análisis.

Con esta notación podemos hacer uso de algoritmos de solución para encontrar los valores dinámicos, planeación de trayectorias y simulaciones, entre otros valores.

En \cite{Corke2007}, se propone un acercamiento sencillo y sistemático para convertir una matriz de transformación homogenea como la de la ecuación \ref{eq:homogeneustransformationmatrix} en los parámetros de Denavit-Hartenberg. 

Al realizar dicho proceso se llegó a los resultados que se muestran en la tabla \ref{table:denavithartenberg}.

\begin{table}
\centering
\caption{Parámetros Denavit Hartenberg}
 \label{table:denavithartenberg}
\begin{tabular}{l|l|l|l|l|}
               & $\theta$ [rad] & d [m]    & a [m]   & $\alpha$ [rad]                        \\ 
\hline
Articulación 1 & $\theta_1$              & 0.152    & 0       & $\frac{\pi}{2}$   \\
Articulación 2 & $\theta_2$              & 0        & -0.244       & 0 \\
Articulación 3 & $\theta_3$                  & 0 & -0.213       & 0                                                  \\
Articulación 4 & $\theta_4$              & -0.012        & 0 & $-\frac{\pi}{2}$   \\
Articulación 5 & $\theta_5$                        & 0.085        & 0 & $\frac{\pi}{2}$  \\
Articulación 6 & $\theta_6$                           & 0        & 0  & $-\frac{\pi}{2}$                                                 
\end{tabular}
\end{table}

Con esto, tenemos una vista simplificada de nuestro brazo robótico, que describe su cinemática directa usando únicamente cuatro parámetros por articulación. El algoritmo desarrollado puede consultarse en el Anexo 2.	

\subsubsection{Cinemática inversa}

Una vez terminada la cinemática directa, toca al turno de abordar la cinemática inversa. Hasta este punto podemos calcular la posición del efector final utilizando las variables de las articulaciones, ahora, procederemos a encontrar las variables de las articulaciones dada una posición y orientación del efector final.

Para nuestro caso, un brazo robótico de seis grados de libertad, la solución tiene doce ecuaciones con seis incognitas, sin embargo, nueve ecuaciones se derivan de la matriz de rotación dentro de la matriz de transformación homogenea ${}_{0}^{6}T$, esto deja unicamente tres ecuaciones independientes, agregando las tres ecuaciones del vector de posición dentro de la matriz de transformación homogenea ${}_{0}^{6}T$ nos da como resultado seis ecuaciones con seis incógnitas.	

Estas ecuaciones son no-lineales y transendentales, lo cual hace más dificil encontrar una solución y como cualquier conjunto de ecuaciones no-lineales, puede existir una sola solución o múltiples soluciones. \cite{Craig2013}

Para resolver el problema de encontrar la cinemática inversa existen dos métodos posibles: \textit{la solución de forma cerrada} y la solución numérica. 

El concenso de la mayoría de los autores, tales como \cite{Spong2005} y \cite{Craig2013} es que se prefiere encontrar la solución de forma cerrada por dos razones principales, la velocidad de solución y la forma en la que se encuentra una solución, es decir, como la cinemática inversa puede tener muchas soluciones se pueden desarrollar reglas para favorecer un tipo de solución con respecto de otra.

Exiten algunos artículos como \cite{Chen2017} en el cuál se ejemplifica la manera de resolver la cinemática inversa de un robot manipulador de seis grados de libertad de forma analítica.

Queda claro que resolver la cinemática inversa no es tarea sencilla, por esta razón y de manera parecida a como se manejó en la sección anterior, se utilizará la librería Robotic Toolbox, la cual incluye un método establecido para calcular la cinemática inversa dada una matriz de transformación homogénea. El algoritmo desarrollado puede consultarse en el Anexo 3.

Así mismo, es importante notar que esta solución se utilizará únicamente en la etapa de desarrollo del brazo robótico y se tendrá que implementar un algoritmo de solución de cinemática inversa en el control final del robot.

\subsection{Cinemática de la velocidad}

Soon.

\subsection{Modelo dinámico}

Hasta ahora, se ha descrito el movimiento del robot sin consideración de las fuerzas y los torques necesarios para producir dicho movimiento, ahora toca el turno de analizarlas.

Como se comentó al principio del capítulo, es necesario establecer una relación entre las posiciones, velocidades y aceleraciones deseadas y el torque o fuerzas que se debe ejercer en los actuadores, esto nos permitirá controlar el brazo robótico así como conocer los parámetros necesarios que los actuadores deben cumplir para satisfacer los requerimientos de velocidad y carga útil.

Para lograrlo, existen dos fomulaciones que se pueden seguir, la formulación Euler-Lagrange o la formulación Newton-Euler. Ambos métodos llegarían a la misma respuesta, sin embargo el camino será diferente.

En la formulación Euler-Lagrange se trata al robot como un todo y se realiza el análisis utilizando la formulación Lagraniana (la diferencia entre energía cinética y energía potencial), por su parte, la formulación Newton-Euler divide el manipulador en cada uno de sus eslabones y describe las escuaciones que definen su movimiento lineal y angular. \cite{Spong2005}

En esta parte de la investigación y como parte de los modelos matemáticos necesarios para seguir diseñando el robot, se utilizará la \textit{formulación Newton-Euler} para obtener los torques y fuerzas necesarias. 

\subsubsection{Formulación Newton-Euler}

Para encontrar los torques a los que deben estar sometidos las articulaciones del brazo robótico se utilizará la herramienta que se ha utilizado a lo largo de este capítulo, con ella, se desarrolla un algoritmo en MATLAB donde se incluiyen los parámetros necesarios para el modelo dinámico tales como masa, centro de gravedad, momento de inercia, fricción viscosa, fricción de Coulomb, inercia del motor y relación de engranes. 

La lista de parámetros para cada articulación y eslabón fue descrita anteriormente en el cuadro \ref{table:parametrosbrazorobotico}.

Cabe destacar que algunos de los parámetros dinámicos son una aproximación optimista, esto debido a no tener aún un diseño final ni todos los componentes seleccionados, sin embargo, es funcional para una primera iteración del torque necesario para continuar con la selección de componentes.

Con los datos del cuadro \ref{table:parametrosbrazorobotico} se selecciona los torques máximos en el escenario más demandante, esto es en una trayectoria en la cuál cada una de las articulaciones se somete a una mayor fuerza.

Con todos los datos correctamente particularizados, obtenemos una lista con todos los valores máximos, se puede apreciar en el cuadro \ref{table:maxtorque}. 

\begin{table}
\centering
\caption{Parámetros del brazo robótico}
\label{table:maxtorque}
\begin{tabular}{l|l|l|l|l|l|l|}
\textbf{Articulaciones}              &  1 & 2 & 3 &  4 &  5 &  6  \\ 
\hline
Torque máximo (N) & 0.2            & 25.35          & 11.94~         & 2.33           & 0.2            & 0.2            
\end{tabular}
\end{table}

El algoritmo realizado está disponible para consulta en el Anexo 4. Así como el repositorio de esta tésis.
       
\section{Componentes eléctricos y electrónicos}

Conforme al análisis realizado en el capítulo anterior es posible empezar la selección de componentes, se empezará por la selección de motores.

\subsection{Selección de motores}

Conforme a lo expuesto en el capítulo \ref{chap:Marcoteorico}, se optó por utilizar motores de corriente continua sin escobillas, conocidos también por sus siglas en inglés BLDC.

De acuerdo con los datos proporcionados por la tabla \ref{table:maxtorque}, la articulación con un torque más demandante es la articulación dos, conocida como el hombro del robot, esta necesita un torque total de 25 N para levantar una carga útil de 2 Kg. Siendo esta la parte más desafiante del brazo robótico, se empezará por el desarrollo de ésta.

Es prácticamente imposible encontrar un motor con ese torque, por lo que será necesario utilizar un sistema de engranes reductores, por ahora, se seleccionará un motor con un torque de salida de entre 1.5 y 2 N para incremenentar el torque de salida a través de engranajes.

Se seleccionó un motor de la marca Turnigy, en particular el modelo Aerodrive SK3 - 6374-149KV, pues según datos experimentales realizados en \cite{odrivedoc} tiene un torque máximo de 3.77 N. Es posible apreciar una imagen de este motor en la figura \ref{fig:sk36374}.

La información experimental del motor SK3 6374 y de los demás motores seleccionados se puede apreciar en la table \ref{table:motordata}.

\begin{figure}
    \centering
    \includegraphics[scale=0.6]{./img/chapter3/sk36374.jpg}
    \caption{Motor a utilizar en la articulación dos.}
    \label{fig:sk36374}
\end{figure}

\begin{table}
\centering
\label{table:motordata}
\caption{Datos de motor SK3 6374}
\begin{tabular}{l|l|l|l|l|l|}
Motor    & Kv    & Corriente máxima & Voltaje máximo & Peso & Torque  \\
         & rpm/V & A                & V              & g    & N $\cdot$ m   \\ 
\hline
SK3 6374 & 149   & 68               & 48             & 840  & 3.77   
\end{tabular}
\end{table}

\subsection{Selección de sensores de posición}
\begin{figure}
    \centering
    \includegraphics[scale=0.3]{./img/chapter3/encoder.png}
    \caption{Odrive v3.5}
    \label{fig:odrive}
\end{figure}


\subsection{Selección de dispositivos de control}

En la selección de un dispositivo de control se tenían los siguientes parámetros:

\begin{itemize}
\itemsep0em
\item Soporte una variedad de voltajes de entrada
\item Soporte una considerable corriente de salida (70-100 A)
\item Control preciso de la posición
\item Poder medir la corriente suministrada al motor
\end{itemize}

Estas especificaciónes fueron cubiertas con la placa ODrive desarrollada por Oskar Weigl. La imagen de la placa se puede encontrar en la figura \ref{fig:odrive}.

\begin{figure}
    \centering
    \includegraphics[scale=0.2]{./img/chapter3/odrive.jpeg}
    \caption{Odrive v3.5}
    \label{fig:odrive}
\end{figure}

\subsection{Selección de fuentes de poder}


\section{Componentes mecánicos}

Una vez determinado los componentes electrónicos necesarios para el correcto funcionamiento del brazo robótico es necesario determinar los componentes mecánicos.

\subsection{Actuadores}

Una de las partes mecánicas más importantes es el actuador, es decir, el mecanismo que proporcionará la fuerza necesaria para mover las articulaciones del robot.

Para la correcta selección o desarrollo del actuador necesitamos dos parámetros que vimos en los capítulos anteriores, el primero es el torque necesario para realizar los desplazamientos del robot a la velocidades y aceleraciones requeridas y el segundo es la velocidad nominal del motor a utilizar.

Por norma general, los motores eléctricos y, para nuestro caso, los motores eléctricos sin escobillas, funcionan a velocidades altas y un torque relativamente pequeño, para esto es necesario utilizar reductores de velocidad, los cuales cumplen dos objetivos primordiales: reducir la velocidad en eje de salida y aumentar el torque en el mismo.

El siguiente paso es escoger un reductor de velocidad que cumpla con los requerimientos específicos de nuestro brazo robótico, los cuales son:

\begin{itemize}
\itemsep0em
\item Alta relación de transmisión, mayor a 10:1.
\item Precisión, necesaria para realizar los movimientos deseados sin demasiado juego. 
\item Efectividad de transmisión, necesaria para no perder demasiada energía.
\end{itemize}

Estos son algunos de los más importantes, existen otros factores a considerar como el tamaño final del reductor y la facilidad de manufactura.

Para el desarrollo de este brazo robótico se han considerado tres diferentes tipos de reductores de velocidad, cada uno con ventajas y desventajas que se mencionaron en la sección \ref{sec:actuadores}. Se decidió optar por un reductor cicloidal, esto debido a su gran relación de transmisión y un juego teórico nulo.  

Para el diseño de este actuador se consultaron diferentes fuentes como \cite{Lei2012} \cite{Nachimowicz2016}, y se utilizó la siguiente formula para el diseño paramétrico en SolidWorks.
\begin{equation}
B_x = r  \cdot \cos(\varphi) - q \cdot \cos(\varphi + \Psi) - e \cdot \cos(N\varphi)
\end{equation}
\begin{equation}
B_y = r  \cdot \sin(\varphi) + q \cdot \sin(\varphi + \Psi) + e \cdot \sin(N\varphi)
\end{equation}

Dónde:

$r$ es la distancia entre el engrane central y los rollers en milímetros.
$q$ es el radio del roller en milímetros.
$e$ es el valor de la excentricidad en milímetros.
$N$ es el número de rollers.
$\Psi$ es el angulo entre $B$ y el punto central del sistema coordenado.

Para calcular $\Psi$ se utiliza la siguiente fórmula:

\begin{equation}
\Psi = \tanh \left[ \frac{\sin((1-N)\varphi)}{\frac{r}{eN}-\cos((1-N)\varphi} \right]
\end{equation}
Desde:
$$
0 \leq \theta \leq 360
$$

Los datos de diseño del engrane cicloidal los podemos apreciar en la tabla \ref{table:valorescicloidal}.

\begin{table}
\centering
\caption{Valores de diseño del engrane cicloidal.}
\label{table:valorescicloidal}
\begin{tabular}{l|l|l|}
\textbf{Descripción} & \textbf{Símbolo} & \textbf{Valor}  \\ 
\hline
Diámetro del engrane & D                & 80 mm.          \\
Diámetro del roller  & d                & 5 mm.           \\
Número de rollers    & N                & 25              \\
Excentricidad        & e                & 1 mm.          
\end{tabular}
\end{table}


\begin{equation}
(38.4 + 1.6) \cdot \cos(2 \cdot \pi \cdot \theta) + 1 \cdot \cos((38.0 + 1.6) \cdot 2 \cdot \pi \cdot \theta / 1.6)
\end{equation}
\begin{equation}
(38.4 + 1.6) \cdot \sin(2 \cdot \pi \cdot \theta) + 1 \cdot \sin((38.0 + 1.6) \cdot 2 \cdot \pi \cdot \theta / 1.6)
\end{equation}

Con la ecuación particulizada utilizamos SOLIDWORKS para crear la pieza que será impresa en plástico ABS por el método de manufactura aditiva que se abordó en -Ingrese sección aquí-.

\begin{figure}
    \centering
    \includegraphics[scale=0.6]{./img/chapter3/cicloidaldisk.png}
    \caption{Engrane cicloidal}
    \label{fig:cicloidaldisk}
\end{figure}

La configuración utilizada en \ref{table:valorescicloidal} nos da como resultado un engrane reductor de 25:1, es decir, en teoría, para el motor SK3 6374 descrito en \ref{table:motordata}, podrámos obtener un torque máximo de 94.25 N $\cdot$ m. 

Como vimos en -ingrese sección- el disco cicloidal es sólo una parte de un reductor cicloidal, para que el mecanismo funcione son necesarias las otras piezas, a continuación se muestra el y el .

El disco, al estar montado con una excentricidad, existe un desbalance de fuerzas cuando se alcanzan altas velocidades, por esta razón, se utilizan dos discos, cada uno montados con un diferencia de 180\degree para alcanzar un balance.





  